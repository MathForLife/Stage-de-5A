\documentclass{report}

\usepackage[utf8]{inputenc}    % Permet d'utiliser les caractères spéciaux
\usepackage[T1]{fontenc}		% Idem

\usepackage[french]{babel} % Utiliser la version française 

\usepackage{layout}  % Permet de visualiser la disposition du texte 

\usepackage[top=2cm, bottom=2cm, left=2cm, right=2cm]{geometry} % Modifie les différentes marges 

\usepackage{setspace} % Augmente les interlignes dans le document

\usepackage{mathtools,amsfonts} % Pour tout ce qui est outils maths + texte math

\usepackage{graphicx}

\pagestyle{headings} % Définit les pieds de page et les en-tête 
% plain = numéro de page en bas, 
% headings = numéro de page + titre chapitre en haut
% empty = vide
\graphicspath{{../Images/}}
%%%%%%%%%%%%%%%%%%%%%%%%%%%%%%%%%%%%%%%%%%%%%%%%%%%%%%
% Commande pour afficher Figure avec légende
\newcommand{\image}[3][0.8]{
\begin{figure}[h]
\centering
\includegraphics[width=#1\textwidth,height=#1\textheight,keepaspectratio]{#2}
\caption{#3}
\label{Fig-#2}
\end{figure}}

%%%%%%%%%%%%%%%%%%%%%%%%%%%%%%%%%%%%%%%%%%%%%%%%%%%%%
\title{Etude d'algorithmes de segmentation appliqués au cadre médical}
\author{Robin CREMESE}
\date{\today}
\begin{document}
\layout
\part{Introduction}
\label{P-Introduction}
\chapter{Présentation du projet}
\label{C-Présentation du projet}
La problématique présentée par l'entreprise Sophia-Genetics à l'équipe du laboratoire de maths de Bordeaux est de trouver des algorithmes performants pour une segmentation binaire d'images 3D issues de scanners et d'IRM de poumons.

Pour se faire, plusieurs axes ont étés abordés, à commencer par une étude préliminaire des différents  algorithme de segmentation basés sur l'évolution de contours. L'étude s'est ensuite poursuivie sur l'utilisation d'histogrammes pour segmenter les images...

\chapter{Etude préliminaire des algorithmes}
\label{C-Etude préliminaire des algorithmes}
Dans ce chapitre nous nous intéresserons aux méthodes de bases par contour actif utilisées pour résoudre notre problème de segmentation.

L'idée principale est de faire évoluer un opérateur de masquage $u_b:\Omega \rightarrow \{0,1\}$ sur une image en définissant une fonctionnelle dépendante des couleurs moyennes des deux régions à segmenter et qu'on cherchera à minimiser.
\section{Notations}
\label{S-Notations}
Afin d'être consistant dans toute la suite de ce rapport, nous utiliserons les notations suivante pour décrire les objets mathématiques utilisés :
\begin{itemize}
	\item[*] L'image, qu'elle soit 2D ou 3D, est vue comme comme une fonction $I$ de 2 ou 3 variables allant d'un espace $\Omega \subset \mathbb{R}^d \text{ , avec }d \in \{2,3\}$ à valeur dans l'intervalle [0,1].
	\item[*] La fonction de masquage est une fonction binaire $u_b:\Omega \rightarrow \{0,1\}$ qui, à chaque pixel de l'image, indique s'il fait partie de la région à segmenter ou non.
	\item[*] On peut ainsi définir la région que l'on souhaite segmenter comme $\Omega_1=\{x\in \Omega ~|~ u_b(x)=1\}$ et le fond de l'image comme $\Omega_0=\Omega \setminus \Omega_1$.
	\item[*] La fonctionnelle que l'on cherchera à faire décroitre, aussi appelée fonction coût sera notée J et elle s'écrira comme la somme de 2 termes : 
	\begin{itemize}
		\item[-] Un terme d'attache aux données
		\item[-] Un terme de régularisation de la solution		
	\end{itemize}
	Le paramètre servant à équilibrer l'influence de chaque terme sur la fonctionnelle $J$ sera appelé le seuil et sera noté $\lambda$.
	\item[*] On note de manière générale $c_i$ la couleur correspondant à la région i de l'image. Dans le cas de notre segmentation binaire $i\in \{0,1\}$ et $c_i$ peut être vue comme la moyenne des couleurs au sein de la région i. 
\end{itemize}
\section{Présentation des méthodes}
\label{S-Présentation des méthodes}
\subsection{Méthode de Chan-Vese}
\label{S-Méthode de Chan-Vese}
Le principe de cette méthode est de définir les différentes régions du masque comme la composée de la fonction de Heaviside avec une fonction de distance signée, appelée fonction Level-Set et notée $\phi$.\\
\underline{Remarque :} Pour rappel, la fonction de Heaviside est définie par $H:x\in \mathbb{R}\rightarrow \begin{cases} 1 \text{ si } x\geq 0 \\ 0 \text{ sinon}\end{cases}$.

\\
Un exemple de fonction Level-Set est donnée à la figure \cite{•} où est calculé la distance 

La fonctionnelle J qu'on cherchera à minimiser comportera différents termes.  
Elle s'écrit donc de la manière suivante :
\begin{equation}
J(u)=\int_{\Omega}||\nabla H(\phi(x))||_{\epsilon}dx+\lambda\Big[\int_{\Omega}|I(x)-c_1|^2H(\phi(x))dx+
\int_{\Omega}|I(x)-c_0|^2(1-H(\phi(x)))dx\Big]
\end{equation}
\subsection{Méthode de Chan-Esedoglu-Nikolova}
\label{SS-Méthode CEN}
On suppose les 2 couleurs $c_1$ et $c_0$ connues, ou du moins bien estimées.

\begin{equation}
\label{Eq-J_CEN}
J(u)=\int_{\Omega}||\nabla u(x)||_{\epsilon}dx+\lambda\Big[\int_{\Omega}|I(x)-c_1|^2u(x)dx+
\int_{\Omega}|I(x)-c_0|^2(1-u(x))dx\Big]
\end{equation}
L'équation qui définie notre algorithme est :
\begin{equation}
\label{Eq-CEN}
	u_{k+1} =P_{\mathcal{A}}\Bigg(u_k+\tau \bigg(div\bigg(\frac{\nabla u_k}
	{||\nabla u_k||_{\epsilon}}\bigg)-\lambda \big[(I-c_1)cv ^2-(I-c_0)^2\big]\bigg)\Bigg)\\
\end{equation}
\subsection{Méthode de Chambol-Pock}
\label{SS-Méthode CP}
La fonctionnelle dont on cherche un point selle est :

\begin{eqnarray}
\label{Eq-J_CP}
		J(u,z)& =&\int_{\Omega}\nabla u(x)\cdot \textbf{z}dx+\lambda\Big[\int_{\Omega}|I(x)-c_1|^2u(x)dx+
\int_{\Omega}|I(x)-c_0|^2(1-u(x))dx\Big]\\
	&=& \int_{\Omega}u(x)div(\textbf{z})dx+\lambda\Big[\int_{\Omega}|I(x)-c_1|^2u(x)dx+
\int_{\Omega}|I(x)-c_0|^2(1-u(x))dx\Big]
\end{eqnarray}


L'équation qui définie notre algorithme est :
\begin{equation}
\label{Eq-CP}
	\begin{cases}
	z_{k+1} = P_{\mathcal{B}}(z_k+\tau_z \nabla \tilde{u}_k)\\
	
	u_{k+1} =P_{\mathcal{A}}\Big(u_k+\tau_u \big(div(z_{k+1})-\lambda \big[(I-c_1)^2-(I-c_2)^2\big]\big)\Big)\\

	\tilde{u}_{k+1}=u_{k+1}+\theta (u_{k+1}-u_k)
	\end{cases}
\end{equation}
\section{Résultats de segmentations}
\label{S-Résultats de segmentations}
On affiche ici les résultat de la segmentation de 3 images par nos 2 algorithmes
\image[1]{Square}{Bonjour}
\end{document}
