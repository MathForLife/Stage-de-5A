\documentclass{report}

\usepackage[utf8]{inputenc}    % Permet d'utiliser les caractères spéciaux
\usepackage[T1]{fontenc}		% Idem

\usepackage[french]{babel} % Utiliser la version française 

\usepackage{layout}  % Permet de visualiser la disposition du texte 

\usepackage[top=2cm, bottom=2cm, left=2cm, right=2cm]{geometry} % Modifie les différentes marges 

\usepackage{setspace} % Augmente les interlignes dans le document

\usepackage{mathtools} % Pour tout ce qui est outils maths

\usepackage{graphicx}

\pagestyle{headings} % Définit les pieds de page et les en-tête 
% plain = numéro de page en bas, 
% headings = numéro de page + titre chapitre en haut
% empty = vide
\graphicspath{{../Images/}}
%%%%%%%%%%%%%%%%%%%%%%%%%%%%%%%%%%%%%%%%%%%%%%%%%%%%%%
% Commande pour afficher Figure avec légende
\newcommand{\image}[3][0.8]{
\begin{figure}[h]
\centering
\includegraphics[width=#1\textwidth,height=#1\textheight,keepaspectratio]{#2}
\caption{#3}
\label{Fig-#2}
\end{figure}}

%%%%%%%%%%%%%%%%%%%%%%%%%%%%%%%%%%%%%%%%%%%%%%%%%%%%%
\title{Etude d'algorithmes de segmentation appliqués au cadre médical}
\author{Robin CREMESE}
\date{\today}
\begin{document}
\layout
\part{Etude préliminaire des algorithmes}
\label{P-Etude préliminaire des algorithmes}
\chapter{Trouver un nom au chapitre}
\section{Présentation des méthodes}
\label{S-Présentation des méthodes}
\subsection{Méthode de Chan-Esedoglu-Nikolova}
\label{SS-Méthode CEN}
On suppose les 2 couleurs $c_1$ et $c_2$ connues, ou du moins bien estimées.

La fonctionnelle qu'on cherche à minimiser est :
\begin{equation}
\label{Eq-J_CEN}
J(u)=\int_{\Omega}||\nabla u(x)||_{\epsilon}dx+\lambda\Big[\int_{\Omega}|I(x)-c_1|^2u(x)dx+
\int_{\Omega}|I(x)-c_2|^2(1-u(x))dx\Big]
\end{equation}
L'équation qui définie notre algorithme est :
\begin{equation}
\label{Eq-CEN}
	u_{k+1} =P_{\mathcal{A}}\Bigg(u_k+\tau \bigg(div\bigg(\frac{\nabla u_k}
	{||\nabla u_k||_{\epsilon}}\bigg)-\lambda \big[(I-c_1)cv ^2-(I-c_2)^2\big]\bigg)\Bigg)\\
\end{equation}
\subsection{Méthode de Chambol-Pock}
\label{SS-Méthode CP}
La fonctionnelle dont on cherche un point selle est :

\begin{eqnarray}
\label{Eq-J_CP}
		J(u,z)& =&\int_{\Omega}\nabla u(x)\cdot \textbf{z}dx+\lambda\Big[\int_{\Omega}|I(x)-c_1|^2u(x)dx+
\int_{\Omega}|I(x)-c_2|^2(1-u(x))dx\Big]\\
	&=& \int_{\Omega}u(x)div(\textbf{z})dx+\lambda\Big[\int_{\Omega}|I(x)-c_1|^2u(x)dx+
\int_{\Omega}|I(x)-c_2|^2(1-u(x))dx\Big]
\end{eqnarray}


L'équation qui définie notre algorithme est :
\begin{equation}
\label{Eq-CP}
	\begin{cases}
	z_{k+1} = P_{\mathcal{B}}(z_k+\tau_z \nabla \tilde{u}_k)\\
	
	u_{k+1} =P_{\mathcal{A}}\Big(u_k+\tau_u \big(div(z_{k+1})-\lambda \big[(I-c_1)^2-(I-c_2)^2\big]\big)\Big)\\

	\tilde{u}_{k+1}=u_{k+1}+\theta (u_{k+1}-u_k)
	\end{cases}
\end{equation}
\section{Résultats de segmentations}
\label{S-Résultats de segmentations}
On affiche ici les résultat de la segmentation de 3 images par nos 2 algorithmes
\image[1]{Square}{Bonjour}
\end{document}
